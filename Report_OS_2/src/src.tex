\section{Метод решения}

Для решения системы линейных уравнений применен метод Гаусса с параллельной обработкой строк матрицы. Основная идея заключается в распараллеливании операций исключения элементов в прямом ходе метода Гаусса.

\subsection{Основной алгоритм работы}
\begin{enumerate}
    \item \textbf{Инициализация:} Создание семафора для ограничения количества одновременно работающих потоков
    \item \textbf{Прямой ход:} Приведение матрицы к треугольному виду:
    \begin{itemize}
        \item Выбор главного элемента в текущем столбце
        \item Параллельное исключение элементов в строках ниже текущей
        \item Каждая строка обрабатывается в отдельном потоке
    \end{itemize}
    \item \textbf{Обратный ход:} Последовательное нахождение неизвестных из треугольной системы
    \item \textbf{Завершение:} Освобождение ресурсов и уничтожение семафора
\end{enumerate}

\subsection{Особенности реализации}
Для обеспечения многопоточности использованы средства Windows API. Реализовано ограничение максимального количества одновременно работающих потоков с помощью семафора.

\section{Описание программы}

Программа реализована в модульном стиле и состоит из трех основных компонентов.

\subsection{Модуль main.c}
Реализует основную логику программы:
\begin{itemize}
    \item Обработка аргументов командной строки (-n, -t, -r, -f)
    \item Выделение памяти под матрицу и векторы
    \item Генерация случайной диагонально-доминируемой матрицы
    \item Чтение матрицы из файла (при указании параметра -f)
    \item Вызов функции решения системы уравнений
    \item Замер времени выполнения и вывод результатов
\end{itemize}

\subsection{Модуль gauss.h/gauss.c}
Содержит реализацию метода Гаусса с многопоточностью:
\begin{itemize}
    \item \texttt{gauss\_solve()} - основная функция решения СЛАУ
    \item Параллельное выполнение операций исключения для строк матрицы
    \item Выбор главного элемента для обеспечения устойчивости
    \item Управление потоками с использованием Windows API (CreateThread, WaitForMultipleObjects)
    \item Обработка ошибок вырожденной матрицы
\end{itemize}

\subsection{Модуль threads\_lim.h/threads\_lim.c}
Предоставляет механизм ограничения количества потоков:
\begin{itemize}
    \item \texttt{threads\_lim\_init()} - инициализация семафора
    \item \texttt{threads\_lim\_acquire()} - захват семафора (уменьшение счетчика)
    \item \texttt{threads\_lim\_release()} - освобождение семафора (увеличение счетчика)
    \item \texttt{threads\_lim\_destroy()} - уничтожение семафора
\end{itemize}

\subsection{Структуры данных}
\begin{itemize}
    \item \textbf{RowTask} - структура для передачи параметров в поток:
    \begin{itemize}
        \item Указатели на матрицу A и вектор b
        \item Размер матрицы n
        \item Текущий шаг k и обрабатываемая строка i
    \end{itemize}
\end{itemize}

\subsection{Используемые системные вызовы Windows}
\begin{itemize}
    \item \textbf{Управление потоками:} CreateThread, WaitForMultipleObjects, CloseHandle
    \item \textbf{Синхронизация:} CreateSemaphore, WaitForSingleObject, ReleaseSemaphore
    \item \textbf{Таймер:} QueryPerformanceFrequency, QueryPerformanceCounter
    \item \textbf{Память:} malloc, free, calloc
\end{itemize}

Архитектура программы обеспечивает эффективное распараллеливание вычислительно сложных операций метода Гаусса при сохранении контроля над количеством используемых потоков.

\section{Результаты экспериментов}

\subsection{Зависимость времени выполнения от количества потоков}

\begin{table}[h]
\centering
\begin{tabular}{|c|c|c|}
\hline
\textbf{Количество потоков} & \textbf{Время выполнения, с} & \textbf{Размер матрицы} \\
\hline
1 & 1.848 & 300×300 \\
2 & 1.452 & 300×300 \\
4 & 1.321 & 300×300 \\
8 & 1.278 & 300×300 \\
\hline
\end{tabular}
\caption{Зависимость времени решения СЛАУ от количества потоков}
\label{tab:threads}
\end{table}

\textbf{Комментарий:}
При увеличении количества потоков наблюдается ускорение вычислений. Для матрицы размером 300×300 использование 8 потоков дает ускорение в 1.45 раза по сравнению с последовательной версией. Наибольший прирост производительности наблюдается при переходе от 1 к 2 потокам, дальнейшее увеличение количества потоков дает меньший выигрыш.

\subsection{Зависимость времени выполнения от размера матрицы}

\begin{table}[h]
\centering
\begin{tabular}{|c|c|c|}
\hline
\textbf{Размер матрицы} & \textbf{Время выполнения, с} & \textbf{Количество потоков} \\
\hline
100×100 & 0.156 & 4 \\
200×200 & 0.603 & 4 \\
300×300 & 1.321 & 4 \\
500×500 & 3.541 & 4 \\
\hline
\end{tabular}
\caption{Зависимость времени решения СЛАУ от размера матрицы}
\label{tab:size}
\end{table}

\textbf{Комментарий:}
Время решения растет быстрее чем квадратично с увеличением размера матрицы, что соответствует вычислительной сложности $O(n^3)$ метода Гаусса. При увеличении размера матрицы с 100×100 до 500×500 время выполнения возрастает примерно в 23 раза, что демонстрирует кубическую зависимость.

\subsection{Эффективность параллелизации}

\begin{table}[h]
\centering
\begin{tabular}{|c|c|c|c|}
\hline
\textbf{Количество потоков} & \textbf{Время, с} & \textbf{Ускорение} & \textbf{Эффективность, \%} \\
\hline
1 & 1.848 & 1.00 & 100.0 \\
2 & 1.452 & 1.27 & 63.5 \\
4 & 1.321 & 1.40 & 35.0 \\
8 & 1.278 & 1.45 & 18.1 \\
\hline
\end{tabular}
\caption{Эффективность параллелизации для матрицы 300×300}
\label{tab:efficiency}
\end{table}

\textbf{Комментарий:}
С увеличением количества потоков эффективность параллелизации значительно снижается. Наибольшая эффективность (63.5\%) достигается при использовании 2 потоков. При 8 потоках эффективность падает до 18.1\%, что связано с накладными расходами на синхронизацию, создание потоков и конкуренцией за доступ к общей памяти.

\subsection{Выводы по экспериментам}

\begin{enumerate}
\item \textbf{Оптимальное количество потоков} для решения СЛАУ методом Гаусса составляет 2-4 потока для матрицы размером 300×300.
\item \textbf{Эффективность параллелизации} максимальна при малом количестве потоков и резко снижается при увеличении их числа.
\item \textbf{Время выполнения} демонстрирует кубическую зависимость от размера матрицы, что соответствует теоретической сложности алгоритма Гаусса.
\item \textbf{Наибольшее ускорение} достигается при переходе от последовательной версии к использованию 2 потоков (ускорение 1.27×).
\item \textbf{Механизм ограничения потоков} успешно выполняет свою функцию, предотвращая избыточное использование ресурсов системы.
\end{enumerate}