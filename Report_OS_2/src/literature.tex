\section{Исходная программа}

\begin{lstlisting}[language=C]
#include <stdio.h>
#include <stdlib.h>
#include <string.h>
#include <time.h>
#include <math.h>
#include <windows.h>
#include "gauss.h"

double now_seconds(){
    LARGE_INTEGER f, c;
    QueryPerformanceFrequency(&f);
    QueryPerformanceCounter(&c);
    return (double)c.QuadPart / (double)f.QuadPart;
}

static void usage(const char *p){
    printf("Usage: %s [-n size] [-t max_threads] [-r seed] [-f filename]\n", p);
    printf("Example: %s -n 500 -t 8 -r 123\n", p);
    printf("(Project: Laba2_OS — Parallel Gaussian Elimination)\n");
}

int main(int argc, char **argv){
    int n = 200;
    int max_threads = 4;
    int seed = -1;
    const char *fname = NULL;

    for(int i=1;i<argc;i++){
        if(strcmp(argv[i],"-n")==0 && i+1<argc) n = atoi(argv[++i]);
        else if(strcmp(argv[i],"-t")==0 && i+1<argc) max_threads = atoi(argv[++i]);
        else if(strcmp(argv[i],"-r")==0 && i+1<argc) seed = atoi(argv[++i]);
        else if(strcmp(argv[i],"-f")==0 && i+1<argc) fname = argv[++i];
        else { usage(argv[0]); return 1; }
    }

    double *A = (double*)malloc(sizeof(double)*n*n);
    double *b = (double*)malloc(sizeof(double)*n);
    double *x = (double*)calloc(n, sizeof(double));
    if(!A || !b || !x){ fprintf(stderr,"alloc fail\n"); return 2; }

    if(fname){
        FILE *f = fopen(fname,"r");
        if(!f){ perror("fopen"); return 3; }
        int rn;
        if(fscanf(f,"%d",&rn)!=1){ fprintf(stderr,"bad file\n"); return 4; }
        if(rn != n){ fprintf(stderr,"file n != specified n; using file n\n"); n = rn; }
        for(int i=0;i<n*n;i++) fscanf(f,"%lf",&A[i]);
        for(int i=0;i<n;i++) fscanf(f,"%lf",&b[i]);
        fclose(f);
    } else {
        if(seed>=0) srand(seed); else srand((unsigned)time(NULL));
        for(int i=0;i<n;i++){
            for(int j=0;j<n;j++){
                double v = ((double)rand()/RAND_MAX)*2.0 - 1.0;
                A[i*n + j] = v;
            }
        }
        for(int i=0;i<n;i++){
            double s = 0;
            for(int j=0;j<n;j++) s += fabs(A[i*n + j]);
            A[i*n + i] += s + 1.0;
            b[i] = ((double)rand()/RAND_MAX)*10.0;
        }
    }

    printf("Matrix size: %d x %d, max_threads=%d\n", n, n, max_threads);
    printf("Program will print active thread counts during elimination.\n");

    double t0 = now_seconds();
    int rc = gauss_solve(A,b,x,n,max_threads,1);
    double t1 = now_seconds();

    if(rc!=0){
        fprintf(stderr,"gauss failed (maybe singular)\n");
        return 5;
    }

    printf("Solved in %.6f seconds\n", t1-t0);
    int toprint = (n<10)?n:10;
    printf("x[0..%d]= ", toprint-1);
    for(int i=0;i<toprint;i++) printf("%.6g ", x[i]);
    printf("\n");

    free(A); free(b); free(x);
    return 0;
}
\end{lstlisting}
Листинг 1: \texttt{main.c} - Основная программа, обрабатывающая аргументы командной строки и управляющая решением СЛАУ

\begin{lstlisting}[language=C]
#include <stdio.h>
#include <stdlib.h>
#include <math.h>
#include <string.h>
#include <windows.h>
#include "gauss.h"
#include "threads_lim.h"

typedef struct {
    double *A, *b;
    int n, k, i;
} RowTask;

static DWORD WINAPI row_eliminate(void *arg) {
    RowTask *t = (RowTask*)arg;
    int n = t->n, k = t->k, i = t->i;
    double *A = t->A, *b = t->b;

    double factor = A[i*n + k] / A[k*n + k];
    for (int j = k; j < n; j++)
        A[i*n + j] -= factor * A[k*n + j];
    b[i] -= factor * b[k];

    free(t);
    threads_lim_release();
    return 0;
}

int gauss_solve(double *A, double *b, double *x, int n, int max_threads, int verbose) {
    threads_lim_init(max_threads);

    for (int k = 0; k < n-1; k++) {
        int pivot = k;
        double maxv = fabs(A[k*n + k]);
        for (int i = k+1; i < n; i++) {
            if (fabs(A[i*n + k]) > maxv) {
                maxv = fabs(A[i*n + k]);
                pivot = i;
            }
        }
        if (pivot != k) {
            for (int j = 0; j < n; j++) {
                double tmp = A[k*n + j];
                A[k*n + j] = A[pivot*n + j];
                A[pivot*n + j] = tmp;
            }
            double tmpb = b[k]; b[k] = b[pivot]; b[pivot] = tmpb;
        }

        if (fabs(A[k*n + k]) < 1e-12) {
            threads_lim_destroy();
            return 1;
        }

        int rows = n - (k+1);
        HANDLE *th = malloc(sizeof(HANDLE) * rows);
        if (!th) {
            threads_lim_destroy();
            return 1;
        }

        for (int i = k+1; i < n; i++) {
            threads_lim_acquire();
            RowTask *task = malloc(sizeof(RowTask));
            if (!task) {
                threads_lim_release();
                free(th);
                threads_lim_destroy();
                return 1;
            }
            task->A = A; task->b = b;
            task->n = n; task->k = k; task->i = i;
            
            th[i - (k+1)] = CreateThread(NULL, 0, row_eliminate, task, 0, NULL);
            if (!th[i - (k+1)]) {
                free(task);
                threads_lim_release();
                for (int j = k+1; j < i; j++) {
                    WaitForSingleObject(th[j - (k+1)], INFINITE);
                    CloseHandle(th[j - (k+1)]);
                }
                free(th);
                threads_lim_destroy();
                return 1;
            }
        }

        WaitForMultipleObjects(rows, th, TRUE, INFINITE);
        
        for (int i = 0; i < rows; i++) {
            CloseHandle(th[i]);
        }
        free(th);

        if (verbose && (k % 10 == 0 || k == n-2))
            printf("[step %d/%d] active threads limited to %d\n", k, n, max_threads);
    }

    for (int i = n-1; i >= 0; i--) {
        double sum = b[i];
        for (int j = i+1; j < n; j++)
            sum -= A[i*n + j] * x[j];
        x[i] = sum / A[i*n + i];
    }

    threads_lim_destroy();
    return 0;
}
\end{lstlisting}
Листинг 2: \texttt{gauss.c} - Реализация метода Гаусса с многопоточной обработкой строк

\begin{lstlisting}[language=C]
#include "threads_lim.h"
#include <windows.h>

static HANDLE sem_handle = NULL;

void threads_lim_init(int max_concurrent){
    sem_handle = CreateSemaphore(NULL, max_concurrent, max_concurrent, NULL);
}

void threads_lim_acquire(void){
    WaitForSingleObject(sem_handle, INFINITE);
}

void threads_lim_release(void){
    ReleaseSemaphore(sem_handle, 1, NULL);
}

void threads_lim_destroy(void){
    if(sem_handle) {
        CloseHandle(sem_handle);
        sem_handle = NULL;
    }
}
\end{lstlisting}
Листинг 3: \texttt{threads\textunderscore lim.c} - Реализация ограничения количества потоков с помощью семафора Windows

\begin{lstlisting}[language=C]
#ifndef GAUSS_H
#define GAUSS_H

#ifdef __cplusplus
extern "C" {
#endif

int gauss_solve(double *A, double *b, double *x, int n, int max_threads, int verbose);

#ifdef __cplusplus
}
#endif

#endif
\end{lstlisting}
Листинг 4: \texttt{gauss.h} - Заголовочный файл для функции решения СЛАУ методом Гаусса

\begin{lstlisting}[language=C]
#ifndef THREADS_LIM_H
#define THREADS_LIM_H

#ifdef __cplusplus
extern "C" {
#endif

void threads_lim_init(int max_concurrent);
void threads_lim_acquire(void);
void threads_lim_release(void);
void threads_lim_destroy(void);

#ifdef __cplusplus
}
#endif

#endif
\end{lstlisting}
Листинг 5: \texttt{threads\textunderscore lim.h} - Заголовочный файл для управления ограничением потоков

\section{Системные вызовы Windows}

\begin{verbatim}
<?xml version="1.0" encoding="UTF-8"?>
<procmon><processlist><process>
<ProcessIndex>470</ProcessIndex>
<ProcessId>76880</ProcessId>
<ParentProcessId>14916</ParentProcessId>
<ParentProcessIndex>471</ParentProcessIndex>
<AuthenticationId>00000000:049676a0</AuthenticationId>
<CreateTime>134057275798954287</CreateTime>
<FinishTime>0</FinishTime>
<IsVirtualized>0</IsVirtualized>
<Is64bit>1</Is64bit>
<Integrity>Обязательная метка\Высокий обязательный уровень</Integrity>
<Owner>DARIA-BOOK\Daria</Owner>
<ProcessName>Procmon64.exe</ProcessName>
<ImagePath>C:\Users\Daria\Downloads\ProcessMonitor\Procmon64.exe</ImagePath>
<CommandLine>&quot;C:\Users\Daria\Downloads\ProcessMonitor\Procmon64.exe&quot; </CommandLine>
<CompanyName>Sysinternals - www.sysinternals.com</CompanyName>
<Version>4.01</Version>
<Description>Process Monitor</Description>
<modulelist>
<module>
<Timestamp>134057285970173388</Timestamp>
<BaseAddress>0x5ff60000</BaseAddress>
<Size>155648</Size>
<Path>C:\Program Files\Bonjour\mdnsNSP.dll</Path>
<Version>3,1,0,1</Version>
<Company>Apple Inc.</Company>
<Description>Bonjour Namespace Provider</Description>
</module>
<module>
<Timestamp>134057285970173388</Timestamp>
<BaseAddress>0x7ff677f80000</BaseAddress>
<Size>2232320</Size>
<Path>C:\Users\Daria\Downloads\ProcessMonitor\Procmon64.exe</Path>
<Version>4.01</Version>
<Company>Sysinternals - www.sysinternals.com</Company>
<Description>Process Monitor</Description>
</module>
... (и так далее, весь XML файл полностью)
</modulelist>
</process>
<process>
<ProcessIndex>471</ProcessIndex>
<ProcessId>14916</ProcessId>
<ParentProcessId>66264</ParentProcessId>
<ParentProcessIndex>597</ParentProcessIndex>
<AuthenticationId>00000000:049676be</AuthenticationId>
<CreateTime>134056832338204969</CreateTime>
<FinishTime>0</FinishTime>
<IsVirtualized>0</IsVirtualized>
<Is64bit>1</Is64bit>
<Integrity>Обязательная метка\Средний обязательный уровень</Integrity>
<Owner>DARIA-BOOK\Daria</Owner>
<ProcessName>Explorer.EXE</ProcessName>
<ImagePath>C:\Windows\Explorer.EXE</ImagePath>
<CommandLine>C:\Windows\Explorer.EXE</CommandLine>
<CompanyName>Microsoft Corporation</CompanyName>
<Version>10.0.19041.4522 (WinBuild.160101.0800)</Version>
<Description>Проводник</Description>
<modulelist>
... (все модули Explorer.EXE)
</modulelist>
</process>
<process>
<ProcessIndex>472</ProcessIndex>
<ProcessId>4</ProcessId>
<ParentProcessId>0</ParentProcessId>
<ParentProcessIndex>473</ParentProcessIndex>
<AuthenticationId>00000000:000003e7</AuthenticationId>
<CreateTime>134052960184910276</CreateTime>
<FinishTime>0</FinishTime>
<IsVirtualized>0</IsVirtualized>
<Is64bit>1</Is64bit>
<Integrity>Обязательная метка\Обязательный уровень системы</Integrity>
<Owner>NT AUTHORITY\СИСТЕМА</Owner>
<ProcessName>System</ProcessName>
<ImagePath>System</ImagePath>
<CommandLine></CommandLine>
<CompanyName></CompanyName>
<Version></Version>
<Description></Description>
<modulelist>
... (все модули System)
</modulelist>
</process>
</processlist>
</procmon>
\end{verbatim}
