\section{Выводы}

В ходе выполнения лабораторной работы были успешно достигнуты все поставленные цели и решены основные задачи:

\begin{enumerate}
    \item \textbf{Освоены механизмы управления потоками в Windows:} На практике применены системные вызовы Windows API для создания и управления потоками (\texttt{CreateThread()}, \texttt{WaitForMultipleObjects()}), а также средства синхронизации (\texttt{CreateSemaphore()}, \texttt{WaitForSingleObject()})
    
    \item \textbf{Реализовано параллельное выполнение вычислений:} Организовано эффективное распараллеливание вычислительно сложных операций метода Гаусса, что позволило значительно ускорить решение систем линейных уравнений большой размерности
    
    \item \textbf{Создана система ограничения потоков:} Разработан механизм контроля максимального количества одновременно работающих потоков с использованием семафоров Windows, обеспечивающий оптимальное использование вычислительных ресурсов
    
    \item \textbf{Решены практические проблемы многопоточного программирования:} 
    \begin{itemize}
        \item Обеспечена потокобезопасность при работе с общими данными (матрицей и векторами)
        \item Реализована обработка ошибок вырожденных матриц в многопоточной среде
        \item Организовано корректное освобождение системных ресурсов (handles потоков и семафоров)
        \item Обеспечено измерение времени выполнения с высокой точностью
    \end{itemize}
\end{enumerate}

Работа продемонстрировала эффективность применения многопоточности для решения вычислительно сложных математических задач. Наблюдается значительное ускорение вычислений при увеличении количества потоков, особенно для матриц большой размерности. Однако также выявлено, что для малых матриц накладные расходы на создание и синхронизацию потоков могут превышать выгоду от параллелизации.

Полученные навыки работы с потоками и синхронизацией будут полезны при создании других программ, где нужно выполнять несколько задач одновременно - например, в программах для сложных расчётов, обработки графики или научных исследований.

\pagebreak