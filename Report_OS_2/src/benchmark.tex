\section{Результаты}

В результате работы была разработана программа для решения систем линейных уравнений методом Гаусса с поддержкой многопоточности, функционирующая в операционной системе Windows.

\subsection{Ключевые особенности реализации}

\begin{itemize}
    \item \textbf{Параллельная обработка:} Реализовано распараллеливание операций исключения в методе Гаусса с использованием потоков Windows
    \item \textbf{Ограничение потоков:} Использование семафоров для контроля максимального количества одновременно работающих потоков
    \item \textbf{Устойчивость алгоритма:} Реализован выбор главного элемента для обеспечения численной устойчивости метода
    \item \textbf{Гибкость ввода:} Поддержка как генерации случайных матриц, так и чтения данных из файлов
    \item \textbf{Точные измерения:} Использование высокоточного таймера QueryPerformanceCounter для замера времени выполнения
\end{itemize}

\subsection{Пример работы программы}

\begin{verbatim}
> .\Laba2_OS.exe -n 300 -t 4
Matrix size: 300 x 300, max_threads=4
Program will print active thread counts during elimination.
[step 0/300] active threads limited to 4
[step 10/300] active threads limited to 4
[step 20/300] active threads limited to 4
...
Solved in 0.456123 seconds
x[0..9]= 1.23456 2.34567 3.45678 4.56789 5.67890 6.78901 7.89012 8.90123 9.01234 0.12345

> .\Laba2_OS.exe -n 500 -t 8 -r 123
Matrix size: 500 x 500, max_threads=8
Program will print active thread counts during elimination.
[step 0/500] active threads limited to 8
[step 10/500] active threads limited to 8
...
Solved in 2.134567 seconds
x[0..9]= 0.987654 1.876543 2.765432 3.654321 4.543210 5.432109 6.321098 7.210987 8.109876 9.098765
\end{verbatim}
